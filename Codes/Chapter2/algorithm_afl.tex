% main
%     initialize the fuzzer
%     while fuzzing is not terminated:
%         cull the queue of tests and update the bitmap
%         select the first entity of the queue, as E
%         fuzz(E)

% calibrate:
%     /* Calibrate a new test case. This is done when processing the input directory
%    to warn about flaky or otherwise problematic test cases early on; and when
%    new paths are discovered to detect variable behavior and so on. */

% trimming:
%     /* Trim all new test cases to save cycles when doing deterministic checks. The
%    trimmer uses power-of-two increments somewhere between 1/16 and 1/1024 of
%    file size, to keep the stage short and sweet. */

\begin{algorithm}
    % \DontPrintSemicolon % Some LaTeX compilers require you to use \dontprintsemicolon instead
    \KwIn{\textbf{$in\_dir$}, \textbf{$out\_dir$}, $instrumented$ \textbf{$Target$}}
    initialize fuzzer\;
    \While{fuzzing is not terminated} {
      cull queue and update bitmaps\;
      $Entry \leftarrow q.first\_entry()$\;
      $fuzz\_one(Entry)$\;
    }
    \caption{afl-fuzz}
    \label{algo:afl}
\end{algorithm}