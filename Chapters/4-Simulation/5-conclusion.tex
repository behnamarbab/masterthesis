\clearpage
\section{Concluding remarks}
\label{sec:ch5-conclusion}

% -T: Conclusions: Explain what we did here

In this chapter, we evaluated our fuzzer with two metrics: code coverage, and execution time. Fuzzbench generates a code coverage report in different plots and tables, and we illustrated the gradual performance of the findings in Figure \ref{fig:cov-growth}. In another perspective, Figure \ref{fig:cov-uniq}, the findings suggest the capability of Waffle in finding unique findings, which are expected due to the allowance of ERU-based guidance.

To measure the execution times, histograms of the findings were compared as in \cref{Figure:exe-freetype,Figure:exe-libjpeg,Figure:exe-libpng,Figure:exe-libxml}. Based on the distribution of the figures, we conclude that Waffle generates more time-consuming executions, which was the goal of this thesis.