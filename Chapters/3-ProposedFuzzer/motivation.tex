\section{Motivating example}

% TODO: talk about algorithmic problems, maybe
The intuition behind this thesis comes from the fact that all the programming competitions announce two main resources that are limited for the execution of the submitted program. This means that a program must be run without any compile-time or run-time errors and generate the correct output; and the whole execution is constrained with a specific time limit and memory limit. \cite{manzoor2008common} In some competitions, the judge system lets the competitors read each others' submitted program. If they can find a vulnerability in the program that is exploitable and causes any unexpected results, the hacker receives the score for their successful \textbf{hacking} attempt. \cite{wasik2018survey}

A fuzzer can automate the process of finding vulnerabilities; and targeting the algorithmic problems requires a resource-aware approach for investigating a larger range of problematic inputs.