\section{Introduction}
In this chapter a new fuzzer is introduced, that is capable of finding the vulnerabilities related to (theoritically) any resource's exhaustion. The first section explains a motivating example leading to our proposed fuzzer. The fuzzer is based on AFL and uses the implementation of Memlock for memory usage assessments. For monitoring the resources, we use compile time instrumentation of the target program using LLVM's APIs; we take advantage of \textbf{visiting} APIs that let us keep track of any type of instructions defined for LLVM. As a result, the instructions related to any resource are counted and this information is later used in the fuzzing stage. The vulnerabilities found by our fuzzer are then tested for exploitability. The short-comings and performance expectations of our fuzzer are investigated before we conclude this chapter.

We will call our proposed fuzzer \textbf{Waffle}, which is derived from \textbf{What An Amazing AFL} - WAAAFL! The summary of our contributions are as follows:

\begin{itemize}
    \item A new instrumentation for collecting runtime information about resource usages, i.e. memory and time.
    \item A new fuzz testing algorithm for collectively considering the former features of AFL and Memlock, as well as the features we introduce in Waffle.
\end{itemize}

