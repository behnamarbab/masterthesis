\section{Introduction}
In this chapter, we introduce \textbf{Waffle} in more details. Waffle is a tool capable of finding the vulnerabilities related to (theoretically) any resource exhaustion. The first section explains a motivating example leading to our proposed fuzzer. This fuzzer is based on AFL and extends its implementation. For monitoring the resources, we use compile-time instrumentation of the target program using LLVM's APIs; we take advantage of \textbf{visiting} APIs that let us keep track of any instructions defined for LLVM. As a result, the instructions related to any resource are counted, and this information is later used in the fuzzing stage.

AFL is the state-of-the-art in finding vulnerabilities and as it is amazing to be developed, the name of our fuzzer comes after \textit{WAAAFL}!

\vspace{\baselineskip}

In this chapter we are contributing the following topics:

\begin{itemize}
    \item An implementation of a fuzzer for finding the worst-case scenario in an algebraic problem.
    \item We use the \textbf{visitor} functions, which are not used in previous works, as we are aware of.
    \item A new instrumentation for collecting runtime information about resource usages. In Waffle, we focus on maximising the number of instructions.
    \item A new fuzz testing approach for collectively considering the former features of AFL, as well as the features we introduce in Waffle.
\end{itemize}
