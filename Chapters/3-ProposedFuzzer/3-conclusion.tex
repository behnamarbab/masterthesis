\newpage
\section{Concluding remarks}

This chapter introduced the development of Waffle. Waffle extends AFL in two parts:

\begin{enumerate}
    \item \textbf{Instrumentation}: \textbf{llvm\_mode} directory is responsible for the instrumentation in Waffle. As we explained in this chapter, \texttt{llvm\_mode/waffle-llvm-rt.o.c} contains the recipe for instrumenting the program in the compilation.
    
    AFL only keeps a coverage-bitmap, but in addition to the coverage-finding methodology, Waffle leverages the \textbf{visitor functions} in LLVM for assessing the more resource-exhaustive executions. Visitor functions count the targeted instructions, and Waffle saves the result in an extra \textit{shared\_memory}.

    Waffle counts the instructions in each basic block, and reduces the counted values for saving into the memory. The size of the \textit{shared\_memory} has increased 4x in Waffle.

    \item \textbf{Fuzzing}: Waffle uses the coverage bitmap and the instruction-counter bitmap for emphasizing the more beneficial fuzzing entries. The main changes are in the procedures of the functions \texttt{calibrate\_case} and \texttt{calc\_score()}. Generally speaking, an interesting test-case runs faster, has more code coverage, and executes more instructions from a specified set of instructions.
\end{enumerate}

We also reviewed some of the modifications in the source code to Waffle's project. The project is located on github, in a public repository \cite{wafl_git}.