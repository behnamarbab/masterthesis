\section{Concluding remarks}

In this chapter we described some basics of our work, that inspired us and lets us implement the current version of our thesis, Waffle. The topics covered in this chapter are:

\begin{itemize}
    \item A brief description of the previous inspiring works for our fuzzer, as well as the stages of a fuzz testing procedure.
    \item We split different types of fuzzers into three categories, whitebox, blackbox, and greybox. These fuzzers have different access to the program resources and as a result, the fuzzing approaches were different.
    \item Code coverage and it's applications in fuzz testing is explained.
    \item We dig into the state-of-the-art fuzzer, AFL, and walked over it's stages that are important for our fuzzer.
    \item LLVM and it's applications in AFL, as well as the instrumentations done by AFL are described.
\end{itemize}

Altogether, we come up with a new approach for fuzzing, as we are aware of. These knowledge led us to introduce Waffle, which is explained in more details in the next chapters.