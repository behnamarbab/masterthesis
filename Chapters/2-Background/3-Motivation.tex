% \section{Performance of AFL} \label{sec:2.3}

% To evaluate the performance of a fuzzer and assess the execution of it, the sample \textbf{C} code is implemented. (Listing \ref{lst:sample_vul}) \cite{sample_code_ref}

% \lstinputlisting[language=C++,style=CodeStyle,caption={Sample vulnerable program},label={lst:sample_vul}]{Codes/Chapter2/sample_vul.c}

% The sample program has a \textbf{Heap buffer overflow vulnerability}. If we compile this code with GCC and debugging flags, the vulnerability stays hidden and the program executes without any errors:

% \lstinputlisting[language=bash,style=CommandStyle,label={lst:gcc-sample},caption={Compile the sample program}]{Codes/Chapter2/exec_1.txt}

% One way to detect the prior vulnerability, is to add AddressSanitizer flag for the compilation \cite{address_san}. ASan is a fast memory error detector. This tool uses memory poisoning for the detection of a heap buffer overflow. (You can find more features of this tool in the reference) \cite{serebryany2012addresssanitizer}

% After we provide the ASan flag for the compilation, we face an error with the same input as the previous example (Listing \ref{lst:exec2}):

% \lstinputlisting[language=bash,style=CommandStyle,label={lst:exec2}]{Codes/Chapter2/exec_2.txt}

% The detection of the vulnerabilities or any other exceptions, signals the operating system for a misbehaviour from the program. A fuzzer would need this signal to evaluate the execution of the target program.