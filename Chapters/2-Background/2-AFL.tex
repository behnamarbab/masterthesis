\section{AFL} \label{sec:2.2}

Michal Zalewski initially developed American Fuzzy Lopper. He introduces this open-source project as \say{a security-oriented fuzzer that employs a novel type of compile-time instrumentation and genetic algorithms to automatically discover clean, interesting test inputs that trigger new internal states of the targeted binary. This substantially improves the functional coverage for the fuzzed code. The compact synthesized corpora produced by the tool help seed other, more labor- or resource-intensive testing regimes down the road.} \cite{zalewski2014american}

AFL is designed to perform \textbf{fast} and \textbf{reliable}, and at the same time, benefit from the \textbf{simplicity} and \textbf{chainability} of the fuzzer \cite{about_afl}:

\begin{itemize}
    \item \textbf{Speed:} Avoiding the time-consuming operations and increasing the number of executions over time.
    \item \textbf{Reliability:} AFL takes strategies that are program-agnostic, leveraging only the coverage metrics for more discoveries. This feature helps the fuzzer to perform consistent in finding the vulnerabilities in different programs.
    \item \textbf{Simplicity:} AFL provides different options, helping the users enhance the fuzz testing in a straightforward and meaningful way. 
    \item \textbf{Chainability:} AFL can test any binary which is executable, and is not constrained by the target software. A driver for the target program can connect the binary to the fuzzer.
\end{itemize}

\subsection{LLVM} \label{sub:2.2.1}

% TODO: update llvm information in references 
\say{The LLVM Project is a collection of modular and reusable compiler and toolchain technologies. The LLVM project has multiple components. The core of the project is itself called “LLVM.” This project contains all of the tools, libraries, and header files needed to process intermediate representations and converts them into object files. Tools include an assembler, disassembler, bitcode analyzer, and bitcode optimizer.} \cite{llvm,lattner2004llvm}

\vspace{\baselineskip}

LLVM can be used as a compiler framework, separated into "front-end" and "back-end." The front-end contains the lexers and parsers, and it accepts the source code to a program and returns the \textbf{intermediate representation (IR)} of the program. The back-end converts the IR into machine language.

\vspace{\baselineskip}

For instrumentation, we insert the logging instructions into each basic block of the program in the front-end. \textbf{Clang} is part of the LLVM toolchain for compiling C/C++ source code. By definition, \say{\textbf{clang} is a C, C++, and Objective-C compiler that encompasses preprocessing, parsing, optimization, code generation, assembly, and linking.}\cite{clang} We extend the phases of compilation so that we are injecting the instructions in compilation. 

\vspace{\baselineskip}

LLVM converts an \textbf{IR} of a program into machine language instructions. The structure of the LLVM project is shown in Figure \ref{fig:llvm}:

\begin{figure}[htpb]
    \includegraphics[width=\textwidth]{Chapter2/llvm.png}
    \centering
    \captionsetup{justification=centering}
    \caption{LLVM architecture: A front-end compiler generates the LLVM IR, and then it is converted into machine code \cite{omni_sci}}
    \label{fig:llvm}
\end{figure}

The instrumentation is applied before the IR generation, and the LLVM IR is fed into the LLVM compiler to generate the machine-specific instructions. As our instrumentation does not affect the LLVM IR compilation, we will not investigate the generated IR.


% Instrumentation
\subsection{Instrumentation and coverage measurements}
\label{instrumentation}

Waffle is based on AFL and we are extending the AFL's instrumentation in our work. The goal of using instrumentation for AFL is to differentiate code coverages. There are two techniques for instrumentation in AFL:

\begin{enumerate}
  \item \textit{$llvm\_mode$}: AFL takes the source code and an instrumentation recipe and generates the instrumented binary of the target program.
  \item \textit{$qemu\_mode$}: AFL leverages the QEMU mode to obtain instrumentation output for closed-source binaries. We don't use this mode in this thesis.
\end{enumerate}

In the LLVM recipe, we instantiate the bitmap and assign it to the shared memory for modifications. The remaining instructions for the recipe will be applied on the basic blocks in \textbf{AFLCoverage} module. This module takes effect in compilation of the program before the generation of IR. We can see some of the implementation of this \textbf{pass} in Listing \ref{lst:afl-llvm}:

\lstinputlisting[language=C++,style=CodeStyle,label={lst:afl-llvm},caption={AFLCoverage module}]{Codes/Chapter2/afl_llvm_pass.cpp}

The recipe for instrumentation fills out the coverage bitmap with the hash values of the paths executed. The instructions are as followed:

\begin{lstlisting}[language=C++,style=CodeStyle,label={lst:hash},caption={Select element and update in shared\_mem}]
  cur_location = <COMPILE_TIME_RANDOM>;
  shared_mem[cur_location ^ prev_location]++; 
  prev_location = cur_location >> 1;
\end{lstlisting}

AFL instruments by adding these instructions into basic blocks. First, a random value is assigned to \textit{curr\_location}. Next, it is XORed with the previous location's value, \textit{prev\_location}, and the resulting value is the location on \textit{shared\_mem}, the \textit{coverage bitmap}, which is incremented by one. The third and final instruction is reseting the \textit{prev\_location} to a new value.

\vspace{\baselineskip}

When AFL runs the instrumented program, every time an instrumented basic block is executed, a dedicated location of $shared\_mem$ in the bitmap is incremented. This algorithm recognizes the different paths that AFL runs through. For instance, in figure \ref{fig:instrumentation}, suppose that we have an instrumented program with the random values which is set in compile time. An execution that walks over basic blocks $1\rightarrow2\rightarrow5$ will increase the value of the corresponding locations by 1; for instance, an increment on $shared\_mem[14287 \oplus 23765]$ is applied for the transition of $1\rightarrow2$ and $shared\_mem[7143 \oplus 21689]$ for $2\rightarrow5$. We can see that the paths $1\rightarrow3\rightarrow4\rightarrow5$ and $1\rightarrow3\rightarrow4\rightarrow3\rightarrow4\rightarrow5$ (which contains a loop), set different locations on bitmap.


\begin{figure}[htpb]
    \includegraphics[width=\textwidth]{Chapter2/instrumentation.png}
    \centering
    \captionsetup{justification=centering}
    \caption{Example for instrumented basic blocks}
    \label{fig:instrumentation}
\end{figure}

AFL uses this coverage feature for discovering new inputs with new code coverages.

\subsubsection*{Visitor functions}

\say{Instruction visitors are used when you want to perform different actions for different kinds of instructions without having to use lots of casts and a big switch statement (in your code, that is). \cite{inst_visitor}} 

\lstinputlisting[language=C++,style=CodeStyle,label={lst:visitors},caption={Visitors example}]{Codes/Chapter2/visitor.cpp}

The specified range can be any two iterators, which can be a Module, Function, BasicBlock, Instruction or any other range between two instruction addresses.

\subsection{AFL Fuzz}

\textit{afl-fuzz.c} has the instructions for fuzzing the target instrumented-program. The Algorithm \ref{algo:afl} illustrates a brief pseudocode of the execution of \textit{afl-fuzz}:

% main
%     initialize the fuzzer
%     while fuzzing is not terminated:
%         cull the queue of tests and update the bitmap
%         select the first entity of the queue, as E
%         fuzz(E)

% calibrate:
%     /* Calibrate a new test case. This is done when processing the input directory
%    to warn about flaky or otherwise problematic test cases early on; and when
%    new paths are discovered to detect variable behavior and so on. */

% trimming:
%     /* Trim all new test cases to save cycles when doing deterministic checks. The
%    trimmer uses power-of-two increments somewhere between 1/16 and 1/1024 of
%    file size, to keep the stage short and sweet. */

\begin{algorithm}
    % \DontPrintSemicolon % Some LaTeX compilers require you to use \dontprintsemicolon instead
    \KwIn{\textbf{$in\_dir$}, \textbf{$out\_dir$}, $instrumented$ \textbf{$Target$}}
    initialize fuzzer\;
    \While{fuzzing is not terminated} {
      cull queue and update bitmaps\;
      $Entry \leftarrow q.first\_entry()$\;
      $fuzz\_one(Entry)$\;
    }
    \caption{afl-fuzz}
    \label{algo:afl}
\end{algorithm}

After the environmental initializations, the fuzzing loop continues until receiving a termination signal. In every iteration of the loop, AFL first culls the corpus of the generated entries. This method assigns a \textbf{favor-factor} (Eq \ref{eq:afl_fav_fac}) to each queue\_entry and marks the \textbf{favorite entries}, as they execute faster and the size of the files are smaller than the rest of the corpus. AFL finds a favorable path for \say{having a minimal set of paths that trigger all the bits seen in the bitmap so far, and focus on fuzzing them at the expense of the rest.} \cite{afl_git} 
\begin{equation}
    fav\_factor = e.exec\_time \times e.length
    \label{eq:afl_fav_fac}
\end{equation}

An entry is selected after culling the queue. AFL evolves the input corpus by generating new entries out of the selected input entry - \texttt{fuzz\_one()}. 

\subsubsection{fuzz\_one()}

% fuzz
%     calibrate the test case
%     trim the test cases (if needed)
%     calculate the performance score
%     bitflip
%     interesting values and dictionary usage
%     random havoc

\begin{algorithm}
    \KwIn{\textbf{$queue Entry$}}
    $T \leftarrow Entry.test\_case$\;
    $calibrate(T)$\;
    $trim(T)$\;
    $perf\_score \leftarrow calculate\_score(T)$\;
    \If{$deterministic\_mode$}{
      $deterministic\_stages(T)$\;
    }
    $MX \leftarrow calc\_stages(perf\_score, \mathcal{C})$\;
    \ForEach{element in $ (0, MX)$}{
      $random\_havoc(T)$\;
    }
    \caption{$fuzz\_one$: Fuzz one Entry}
    \label{algo:fuzzone}
\end{algorithm}

Fuzzing a single entry requires the calibration of the test-cases; calibration helps AFL in evaluating the stats of the current entry. This evaluation is stored in \texttt{perf\_score}, and the number of trials for generating new inputs from the \textbf{random\_havoc} stage is calculated using the \texttt{perf\_score}. AFL, as a coverage-based fuzzer, assigns higher performance score for the entries that are executed faster and have bigger bitmaps.

The evolutionary algorithms for generating new entries are applied in two stages: \textbf{deterministic} and \textbf{random\_havoc} stages. As shown in Algorithm \ref{algo:fuzzone}, AFL initially tries the basic, deterministic algorithms. These algorithms are executed for a specific number of times, in the same order, and once for each fuzzing trial. . Bit-flipping, byte-flipping, simple arithmetic operations, using known integers and values from dictionaries, are a sequence of mutations that AFL applies on an entry in \textbf{deterministic} stage. Each one of the above operations tweak a small portion of the fuzzed input, and does not modify the file in large portions - up to 32 bits changes in each tweak.

The \textbf{havoc} stage is a cycle of stacked random tweaks. AFL assesses the current entry to insert into the queue. Each mutation is selected randomly and with a higher \texttt{perf\_score}, this stage continues more fuzzing over the current entry. The \texttt{random\_havoc} stage consists techniques such as bit flips, overwriting with random and interesting integers, block deletion, block duplication, and (if supplied) assorted dictionary-related operations \cite{afl_userguide}. An abstract implementation of the \texttt{havoc\_stage} can be found in Appendix \ref{app:havoc}.

\subsubsection*{calculate\_score()}
\label{sub:calc_score}

This function calculates how much AFL desires to iterate in havoc stage, for the current entry. By default, AFL is interested in fuzzing an input with less execution-time, and simultaneously, showing more coverage, and it's generation depth is higher. The depth of a child entry is one more than the depth of it's parent. For more information, check the following abstraction of the function [Listing \ref{lst:calc_score}]: 

\lstinputlisting[language=C,style=CodeStyle,label={lst:calc_score},caption={An abstract implementation of calculate\_score()}]{Codes/Chapter2/calculate_score.c}

\subsubsection*{common\_fuzz\_stuff()}
The newly fuzzed (generated) inputs must pass \texttt{common\_fuzz\_stuff()} for validating and instantiating a \texttt{queue\_entry}. The validation checks the length of the fuzzed file, executes the program and keeps the exit-value of the execution. In the end, AFL calls \texttt{save\_if\_interesting()} to insert the entry into the queue, if it is interesting. The function \texttt{has\_new\_bits()} considers how interesting an entry is.

\lstinputlisting[language=C,style=CodeStyle,label={lst:common_stuff},caption={An abstract implementation of common\_fuzz\_stuff()}]{Codes/Chapter2/common_fuzz_stuff.c}

\subsubsection*{has\_new\_bits()}
AFL calls this function after each execution of the program, and the optimization of this code participates an important role.

\lstinputlisting[language=C,style=CodeStyle,label={lst:has_new_bits},caption={The implementation of has\_new\_bits()}]{Codes/Chapter2/has_new_bits.c}

\subsubsection*{Status screen}

The \textbf{status screen} is a UI for the status of the fuzzing procedure. As it is shown in Figure \ref{fig:status_screen}, there are multiple stats provided in real-time updates:
    
\begin{figure}[!b]
    \includegraphics[width=\textwidth]{Chapter2/afl_screen.png}
    \centering
    \caption{AFL status screen}
    \label{fig:status_screen}
\end{figure} 

\begin{enumerate}
    \item \textbf{Process timing}: This section tells about how long the fuzzing process is running.
    \item \textbf{Overall results}: A simplified information about the progress of AFL in finding paths, hangs, and crashes. 
    \item \textbf{Cycle progress}: As mentioned before, AFL takes one input and repeats mutating it for a while. This section shows the information about the current cycle that the fuzzer is working on.
    \item \textbf{Map coverage}: \say{The section provides some trivia about the coverage observed by the instrumentation embedded in the target binary. The first line in the box tells you how many branches we have already hit, in proportion to how much the bitmap can hold. The number on the left describes the current input; the one on the right is the entire input corpus's value. The other line deals with the variability in tuple hit counts seen in the binary. In essence, if every taken branch is always taken a fixed number of times for all the inputs we have tried, this will read "1.00". As we manage to trigger other hit counts for every branch, the needle will start to move toward "8.00" (every bit in the 8-bit map hit) but will probably never reach that extreme. 
    
    Together, the values can help compare the coverage of several different fuzzing jobs that rely on the same instrumented binary.
    }
    \item \textbf{Stage progress}: The information about the current mutation stage is briefly provided here.
    \item \textbf{Findings in depth}: The crashes and hangs and any other findings (here we have the other information about the coverage) are presented in this section.
    \item \textbf{Fuzzing strategy yields}: To illustrate more stats about the strategies used since the beginning of fuzzing, and for comparison of those strategies, AFL keeps track of how many paths were explored, in proportion to the number of executions attempted, for each of the fuzzing strategies.
    \item \textbf{Path geometry}: The information about the inputs and their depths, which says how many generations of different paths were produced in the process. For instance, we call the seeds we provided for fuzzing the "level 1" inputs. Next, a new set of inputs is generated as "level 2", the inputs derived from "level 2" are "level 3," and so on.
\end{enumerate}

\subsection{Start Fuzz Testing}

AFL requires the instrumented binary for execution. To start the instrumentation, AFL uses \textit{afl-clang}, which is built with the coverage recipe included. The following command instruments the sample program \ref{lst:sample_vul}:

\begin{lstlisting}[language=bash,style=CommandStyle,caption=Instrument $sample\_vul$.c]
    afl-clang sample_vul.c -o sample_vul_i
\end{lstlisting}

Now AFL can run this program in \textit{afl-fuzz} with the coverage instrumentations.

\begin{lstlisting}[language=bash,style=CommandStyle,caption=Execute AFL]
  # afl-fuzz -i <in_dir> -o <out_dir> [options] -- /path/to/fuzzed/app [params]
  afl-fuzz -i in_dir -o out_dir -- ./sample_vul_i
\end{lstlisting}

The fuzzing continues until the fuzz testing is stopped using a termination signal. Pressing \textit{Ctrl+C} is a common command for this purpose. All of the recorded information are accessed through the output directory \textit{out\_dir}.