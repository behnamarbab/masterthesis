\section{Introduction}
\label{sec:intro}

% -T: Talk about software security testing and it's challenges
% -K: The needs which led to the usage of software
% -K: The daily tasks which are happening because of software
% -K: Explain the importance of keeping the software society safe

Our daily lives are tangled with the vast usage of software in many different aspects. Nowadays, most of the population in the modern world have a mobile phone in their hands, with hundreds or thousands of small to large software installed on their devices. Vehicles use software to monitor the sensors, and react to different situations which help saving and easing many lives. Instead of using mechanical tools and devices, the software can now do impossible calculations and suggest most reliable solutions to different problems affecting our lives. The society communicates easier than before, and a world without the existence of software would be unimaginable, unless we choose less productivity and accept incomparable expenses. In a glance, huge loads of data is transmitted in every second, and the services serve various customers who are using their technology. But the reliance of the humanitiy on software establishes a weakness which is targeted by hackers. Thereby, security researchers and developers are responsible for stabilizing the field of software security.

% -T: Talk about the automation and fuzz testing
% -K: Explain how hard the software testing could happen
% -K: Explain the manual and automatic approaches for testing a software
% -K: Explain the benefit of using automated approaches

The software development has shown to be challenging, as there are unseen mistakes which happen due to wrong approaches, or even worse, basically, because of possible human  errors. The bugs may remain hidden for many years, and critical software must be under investigation, as the hackers are always looking for any vulnerability which helps them get passed the protected areas. To investigate the bugs and prevent their existence, different phases of testing are applied before and after the accomplishment of products. One of the techniques for revealing the bugs, is to monitor the execution of program, and evaluate the correctness of the software's execution. A manual approach would be understanding source code and generated binaries, and pointing out the present deficiencies and faults. This cumbersome task of reading the codes, suggests the development of software security tools to speed-up the testing. Fuzzer, is an \textbf{automated software security testing} tool, which helps researchers find vulnerabilities, supported by the processing power of the computers.