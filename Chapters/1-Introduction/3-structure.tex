\section{Thesis Organization}
\label{sec:1.3}

In Chapter 2, some of the related works in the area of fuzz testing are reviewed; software vulnerabilities and their occurrences are discussed, and a graphical representation of software is explained for understanding the execution of the program and unfolding the causes of vulnerabilities. Next, a classification of fuzzers and their usages are described. We wrap up the second chapter by introducing a fuzzer which is the base of our work. In Chapter 3, the proposed fuzzer is introduced. We explain how the solution is applied to a program with possible vulnerabilities. To wrap up the 3rd chapter, a sample program is tested, and the performance of the fuzz testing tool is analyzed. The performance of the implementations is evaluated in Chapter 4. A frontier benchmarking tool for fuzz testing is introduced, and the reports are presented afterward. We wrap up the thesis and conclude in Chapter 5. In addition, some future works are suggested for continuing this work.